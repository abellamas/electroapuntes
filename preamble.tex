\usepackage[spanish]{babel}

% =====================================================
% Fuente local dentro del proyecto (NO depende del sistema)
% =====================================================
\usepackage{fontspec}
\setmainfont[
  Path = fonts/,
  UprightFont = Lato-Regular.ttf,
  ItalicFont = Lato-Italic.ttf,
  BoldFont = Lato-Bold.ttf,
  BoldItalicFont = Lato-BoldItalic.ttf
]{Lato}

% =====================================================
% Paquetes básicos
% =====================================================
\usepackage{amsmath, amssymb}
\usepackage{graphicx}
\usepackage{hyperref}
\usepackage{geometry}

% =====================================================
% Márgenes tipo libro (A4, doble faz)
% =====================================================
\geometry{
  a4paper,
  inner=3.0cm,
  outer=2.2cm,
  top=2.5cm,
  bottom=2.5cm
}

% =====================================================
% Estilo de párrafos (opcional pero suele quedar mejor para apuntes)
% =====================================================
\setlength{\parindent}{0pt}
\setlength{\parskip}{6pt}

% =====================================================
% Interlineado tipo libro (académico)
% =====================================================
\usepackage{setspace}
\onehalfspacing

% =====================================================
% Encabezados y pies de página estilo libro
% Página par: capítulo | Página impar: título del apunte
% Número de página en el exterior
% =====================================================
\usepackage{fancyhdr}
\pagestyle{fancy}
\fancyhf{}

\fancyhead[LE]{\nouppercase{\leftmark}}
\fancyhead[RO]{\nouppercase{\textit{Título del Apunte}}}

\fancyfoot[LE,RO]{\thepage}

\renewcommand{\headrulewidth}{0.4pt}
\renewcommand{\footrulewidth}{0pt}
\setlength{\headheight}{14.5pt}

% Quitar "Capítulo X" del encabezado (deja solo el nombre)
\renewcommand{\chaptermark}[1]{\markboth{#1}{}}
\renewcommand{\sectionmark}[1]{\markright{#1}}

% =====================================================
% Estilo visual de capítulos (más libro/editorial)
% =====================================================
\usepackage{titlesec}
\titleformat{\chapter}
  {\normalfont\huge\bfseries}
  {\thechapter.}
  {1em}
  {}
\titlespacing*{\chapter}
  {0pt}
  {3.5cm}
  {1.5cm}

% =====================================================
% Ajustes académicos: captions + ecuaciones por capítulo
% =====================================================
\usepackage{caption}
\captionsetup{
  font=small,
  labelfont=bf
}

\numberwithin{equation}{chapter}

% =====================================================
% Hyperref sobrio (links negros)
% =====================================================
\hypersetup{
  colorlinks=true,
  linkcolor=black,
  citecolor=black,
  urlcolor=black
}
