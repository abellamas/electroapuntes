\section{Altoparlante}
\[
\int_0^\infty e^{-x^2}\,dx = \frac{\sqrt{\pi}}{2}
\]
\section{Altoparlante al aire libre}


Un altoparlante al aire libre se comporta en un principio como un dipolo acústico. Para el análisis se toma como convención que la cara frontal actúa comprimiendo el aire, lo cual se denominará polaridad positiva, mientras que la cara trasera en el mismo instante está realizando una expansión del aire, denominandose esto polaridad negativa. En este modelo hay que tener en cuenta que existe una diferencia de caminos $b$ entre la cara trasera y delantera del parlante, que no necesariamente es la distancia entre ambas caras; esto generará un defasaje entre la señal acústica de la cara trasera y delantera medida a una distancia $r$ del centro del dipolo, tal como se ve en la figura \ref{fig:dipole_scheme}

\begin{figure}[h]
\centering
\includegraphics[width=0.7\linewidth]{figures/cap2/dipole_scheme.png}
\caption{Esquema geométrico de un dipolo acústico}
\label{fig:dipole_scheme}
\end{figure}



\section{Altoparlante en bafle infinito}
Si querés probar una figura, poné un archivo en \texttt{figures/} y usá:

\begin{figure}[h]
\centering
% \includegraphics[width=0.7\linewidth]{figures/ejemplo.png}
\caption{Ejemplo de figura (descomentá includegraphics cuando tengas imagen)}
\end{figure}
